\documentclass[
dvipsnames, table,   % xcolor option
format=acmsmall,     % paper format
anonymous=true,      % Whether to make author(s) anonymous - used with review below to make anon
authorversion=false, % Whether to generate a special version for the authors’ personal use or posting
]{acmart}

% author-year citation style (required by TELO)
\citestyle{acmauthoryear}

% algorithms
\usepackage{algcompatible, algorithm}
\usepackage[noend]{algpseudocode}
\usepackage{import}
% spacing after full stops in \eg, \ie
\usepackage{xspace}

% subcaptions for side-by-side algorithms
\usepackage[]{subcaption}
\DeclareCaptionSubType*{algorithm}
\renewcommand\thesubalgorithm{\thealgorithm\alph{subalgorithm}}

% table and siunits
\usepackage{tabularx, siunitx}

% tikzm
\usepackage{tikz}
\usetikzlibrary{arrows, decorations.markings}


%%%%%%%%%%%%%%%%%%%%%%%%%%%%%%%%% names and commands
% names for the techniques
\newcommand\eFront{$\epsilon$-PF\xspace}
\newcommand\eRandom{$\epsilon$-RS\xspace}

% best technique and join best labels
%\newcommand{\best}{\color{red}}
%\newcommand{\statsimilar}{\color{blue}}
\newcommand{\best}{{\cellcolor[gray]{0.75}}}
\newcommand{\statsimilar}{{\cellcolor[gray]{0.9}}}

\newcommand\mX{\mathcal{X}}
\newcommand\Real{\mathbb{R}}
\newcommand\mF{\mathcal{F}}
\newcommand\mP{\mathcal{P}}
\newcommand\Papprox{\tilde{\mathcal{P}}}
\newcommand\mGPs{$\mathcal{GP}$s\xspace}
\newcommand\mGP{\ensuremath{\mathcal{GP}}\xspace}
\newcommand\mf{\mathbf{f}}
\newcommand\mD{\mathcal{D}}
\newcommand\mN{\mathcal{N}}
\newcommand\normal{\mathcal{N}}
\newcommand\mS{\mathcal{S}}
\newcommand\EI{\alpha_{EI}}
\newcommand\WEI{\alpha_{WEI}}
\newcommand\qEI{\alpha_{qEI}}
\newcommand\PI{\alpha_{PI}}
\newcommand\UCB{\alpha_{UCB}}
\newcommand\LCB{\alpha_{LCB}}
\newcommand\async{\alpha_{async}}
\newcommand\prob{p}
\newcommand{\inv}{^{-1}}
\newcommand\natnum{\mathbb{N}}
\newcommand\expc{\mathbb{E}}
\newcommand*{\medcup}{\mathbin{\scalebox{1.5}{\ensuremath{\cup}}}}
\DeclareMathOperator*{\union}{\medcup}
\DeclareMathOperator*{\argmax}{\arg\!\max}
\DeclareMathOperator*{\argmin}{\arg\!\min}
\DeclareMathOperator*{\bigO}{\mathcal{O}}
\DeclareMathOperator*{\erf}{\text{erf}}
\DeclareMathOperator*{\cov}{\text{cov}}
\DeclareMathOperator{\diag}{diag}
\DeclareMathOperator{\nondom}{nondom}
\DeclareMathOperator{\LatinHypercubeSampling}{LatinHypercubeSampling}

\newcommand{\trp}{^\top}
\newcommand{\given}{\,|\,}
\newcommand{\bx}{\mathbf{x}}
\newcommand{\bX}{\mathbf{X}}
\newcommand{\by}{\mathbf{y}}
\newcommand{\bz}{\mathbf{z}}
\newcommand{\brr}{\mathbf{r}}
\newcommand{\bff}{\mathbf{f}}
\newcommand{\bF}{\mathbf{F}}
\newcommand{\bzero}{\mathbf{0}}
\newcommand{\fhat}{\hat{f}}
\newcommand{\fstar}{f^\star}
\newcommand{\xnext}{\bx'}
\newcommand{\Xnext}{X'}
\newcommand{\Xcand}{X_c}
\newcommand{\Xq}{X_{q}}
\newcommand{\tX}{\tilde{X}}
\newcommand{\tSigma}{\tilde{\Sigma}}
\newcommand{\bkappa}{\boldsymbol{\kappa}}
\newcommand{\bSigma}{\boldsymbol{\Sigma}}
\newcommand{\bmu}{\boldsymbol{\mu}}
\newcommand{\FIXME}[1]{\textcolor{red}{[\textbf{FIXME} \textsl{#1}]}}
\newcommand{\mnote}[2][\textcolor{red}{\dagger}]{$#1$\marginpar{\color{red}\raggedright\tiny$#1$ #2}}

\newcommand*{\eg}{e.g.\@\xspace}
\newcommand*{\ie}{i.e.\@\xspace}
\newcommand*{\etal}{\textit{et al.}\@\xspace}

%%%%%%%%% ACM STUFF
%% Rights management information.  This information is sent to you
%% when you complete the rights form.  These commands have SAMPLE
%% values in them; it is your responsibility as an author to replace
%% the commands and values with those provided to you when you
%% complete the rights form.
\setcopyright{acmcopyright}
\copyrightyear{2020}
\acmYear{2020}
%\acmDOI{10.1145/1122445.1122456}

%%
%% These commands are for a JOURNAL article.
\acmJournal{TELO}
\acmVolume{1}
\acmNumber{1}
\acmArticle{1}
\acmMonth{1}

%%
%% end of the preamble, start of the body of the document source.
\begin{document}

\section{Dominated Hypervolume measurement}

\begin{table}[h]
\setlength{\tabcolsep}{2pt}
\sisetup{table-format=1.2e-1,table-number-alignment=center}
\resizebox{\textwidth}{!}{%
\import{tables/}{hv_table_1.tex}}
\caption{Median Dominated Hypervolumes after 250 evaluations, for 31 repeated optimsaitons from randomised starting conditions. Best performance for each problem is highlighted in dark grey, and those which are
statistically equivalent to the best method according to a one-sided paired Wilcoxon signed-rank test \citep{knowles:testing} with Holm-Bonferroni correction
\citep{holm:test} ($p\geq0.05$), are shown in light grey.}
\end{table}


\begin{table}[h]
\setlength{\tabcolsep}{2pt}
\sisetup{table-format=1.2e-1,table-number-alignment=center}
\resizebox{\textwidth}{!}{%
\import{tables/}{hv_table_2.tex}}
\caption{Median Dominated Hypervolumes after 250 evaluations, for 31 repeated optimsaitons from randomised starting conditions. Best performance for each problem is highlighted in dark grey, and those which are
statistically equivalent to the best method according to a one-sided paired Wilcoxon signed-rank test \citep{knowles:testing} with Holm-Bonferroni correction
\citep{holm:test} ($p\geq0.05$), are shown in light grey.}
\end{table}


\begin{table}[h]
\setlength{\tabcolsep}{2pt}
\sisetup{table-format=1.2e-1,table-number-alignment=center}
\resizebox{\textwidth}{!}{%
\import{tables/}{hv_table_3.tex}}
\caption{Median Dominated Hypervolumes after 250 evaluations, for 31 repeated optimsaitons from randomised starting conditions. Best performance for each problem is highlighted in dark grey, and those which are
statistically equivalent to the best method according to a one-sided paired Wilcoxon signed-rank test \citep{knowles:testing} with Holm-Bonferroni correction
\citep{holm:test} ($p\geq0.05$), are shown in light grey.}
\end{table}

\clearpage
\section{igd+ measurements}

\begin{table}[h]
\setlength{\tabcolsep}{2pt}
\sisetup{table-format=1.2e-1,table-number-alignment=center}
\resizebox{\textwidth}{!}{%
\import{tables/}{igd_table_1.tex}}
\caption{some caption}
\end{table}


\begin{table}[h]
\setlength{\tabcolsep}{2pt}
\sisetup{table-format=1.2e-1,table-number-alignment=center}
\resizebox{\textwidth}{!}{%
\import{tables/}{igd_table_2.tex}}
\caption{some caption}
\end{table}


\begin{table}[h]
\setlength{\tabcolsep}{2pt}
\sisetup{table-format=1.2e-1,table-number-alignment=center}
\resizebox{\textwidth}{!}{%
\import{tables/}{igd_table_3.tex}}
\caption{some caption}
\end{table}
% Table~\ref{tbl:synthetic_results} shows the median regret, \ie the median
%difference between the estimated optimum $\fstar$ and the true optimum %,
%over the 51 repeated experiments, together with the median
%absolute deviation from the median (MAD). The method with the lowest (best)


\end{document}
%%% Local Variables:
%%% mode: latex
%%% TeX-master: t
%%% auto-fill-mode: t
%%% End:
